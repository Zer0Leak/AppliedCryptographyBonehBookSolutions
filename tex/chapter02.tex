\chapter{Encyption}

\noindent
\textbf{2.1}
\textbf{(multiplicative one-time pad).} We may also define a “multiplication mod p” variation of the one-time pad.
This is a cipher E = (E, D), defined over (K, M, C), where K := M := C := {1, . . . , p-1}, where p is a prime.
Encryption and decryption are defined as follows:
\[
E(k, m) := k \cdot m \bmod p \quad\quad
D(k, c) := k^{-1} \cdot c \bmod p
\]
Here, $k^{-1}$ denotes the multiplicative inverse of k modulo p. Verify the correctness property for this
cipher and prove that it is perfectly secure.


\begin{tcolorbox}[solutionbox, title=Answer: Auxiliary]
    Notice that $k^{-1} \cdot k \bmod p = 1$ (multiplicative inverse of $k$ modulo $p$)
    \vspace{1em}

    Given $p$ is prime, $k^{-1}$ is unique. Let's prove.

    Suppose there is $x$ and $y$, such that $k \cdot x \bmod p = 1 = k \cdot y \bmod p$. Also, make sure $x$ and $y$ are reduced by $\bmod p$. \textit{i.e.} $0 < x, y < p$

    Since $p$ is prime and $0 < k < p$, we can divide both sides by $k$.

    $x \bmod p = y \bmod p$

    $x = y$

\end{tcolorbox}


\begin{tcolorbox}[solutionbox, title=Answer: Correctess]
    Notice that $k^{-1} \cdot k \bmod p = 1$ (multiplicative inverse of k modulo p)
    \vspace{1em}

    $c = Enc(k, m):= k \cdot m \bmod p$

    $Dec(k, Enc(k, m)) = k^{-1} \cdot c \bmod p$

    $Dec(k, Enc(k, m)) = k^{-1} \cdot (k \cdot m \bmod p) \bmod p$

    $Dec(k, Enc(k, m)) = k^{-1} \cdot k \cdot m \bmod p \bmod p$

    $Dec(k, Enc(k, m)) = k^{-1} \cdot k \cdot m \bmod p$

    $Dec(k, Enc(k, m)) = 1 \cdot m \bmod p$

    Since $1 \leqslant  m \leqslant p-1$, then

    $Dec(k, Enc(k, m)) = m$
\end{tcolorbox}

\begin{tcolorbox}[solutionbox, title=Answer: Perfectly Secure]
    $m^{-1} \cdot m \bmod p = 1$ (multiplicative inverse of m modulo p)
    \vspace{1em}

    $k \cdot m \bmod p = c$

    $k \cdot m \cdot m^{-1} \bmod p = c \cdot m^{-1} \bmod p$

    $k \bmod p = c \cdot m^{-1} \bmod p$

    Since $m^{-1}$ is unique, for every $c \in \mathcal{C}$, and for all message $m \in \mathcal{M}$

    $N_c = |\{k \in \mathcal{K}: E(k,m) = c\}| = 1$

    This is perfectly secure according to \textbf{Theorem 2.1 (ii)}
\end{tcolorbox}

\newpage
\noindent
\textbf{2.2}
\textbf{(A good substitution cipher).} Consider a variant of the substitution cipher $\mathcal{E} = (E, D)$
defined in Example 2.3 where every symbol of the message is encrypted using an independent
permutation. That is, let $\mathcal{M} = \mathcal{C} = \Sigma^L$  for some a finite alphabet of symbols $\Sigma$ and some L. Let
the key space be $\mathcal{K} = S^L$ where $S$ is the set of all permutations on $\Sigma$. The encryption algorithm
E(k, m) is defined as:
\[
E(k, m) := k[0](m[0]), k[1](m[1]), . . . , k[L-1](m[L-1])
\]
Show that $\mathcal{E}$ is perfectly secure.


\begin{tcolorbox}[solutionbox, title=Answer]
    The encryption decryption of each symbol is independent. At each index there is an independent \textbf{substitution cipher}.
    \vspace{1em}

    Therefore, we can reduce to prove that $\mathcal{M} = \mathcal{C} = \Sigma$, and $\mathcal{K} = S$, \textit{i.e.}, $m$ and $c$ has length 1, and $|\mathcal{K}| = |\Sigma|!$ is perfectly secure.
    \vspace{1em}

    $P_r[Enc(k, m) = c] = P_r[k(m) = c] = 1/|\Sigma|$ for all $m \in \mathcal{M}$ and all $c \in \mathcal{C}$ and any distribution of $\mathcal{K}$
    \vspace{1em}

    Therefore it is perfectly secure directly from the \textbf{Definition 2.1 (perfect security)} 

\end{tcolorbox}



% $\bigconcat_{i=1}^n m_i$

\newpage
\noindent
\textbf{2.3}
\textbf{(A broken one-time pad).} Consider a variant of the one time pad with message space
$\{0, 1\}^L$ where the key space $\mathcal{K}$ is restricted to all $L$-bit strings with an even number of 1’s. Give an
efficient adversary whose semantic security advantage is 1


\begin{tcolorbox}[solutionbox, title=Answer]
    The adversary, $\mathcal{A}$, choose $m_0 := 0^L$, and $m_1 := 0^{L-1}1$
    \vspace{1em}

    If the cipher text $c$ has an even parity it outputs $\hat{b} = 0$ (because it was exactly the parity of the key)
    \vspace{1em}

    Otherwise, cipher text $c$ has an odd parity, it outputs $\hat{b} = 1$. Because the number of 1's will be the number of 1's in the key, that is even, minus one if the key has a 1 at index $L-1$, or plus one, if the key has a 0 at index $L-1$.

\end{tcolorbox}
